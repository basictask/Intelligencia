% Settings for the default beamer theme
\documentclass[english, aspectratio=169]{beamer}
\usepackage[T1]{fontenc}
\usepackage[utf8]{inputenc}
\usepackage{tabularx}
\usepackage{babel}
\setcounter{secnumdepth}{3}
\setcounter{tocdepth}{3}

\makeatletter

\newcommand\makebeamertitle{\frame{\maketitle}}

% (ERT) argument for the TOC
\AtBeginDocument{%
  \let\origtableofcontents=\tableofcontents
  \def\tableofcontents{\@ifnextchar[{\origtableofcontents}{\gobbletableofcontents}}
  \def\gobbletableofcontents#1{\origtableofcontents}
}

% Theme settings
\usetheme{Frankfurt}
\usecolortheme{default}
\usefonttheme[onlymath]{serif}

% Template settings
\setbeamertemplate{navigation symbols}{}
\setbeamertemplate{blocks}[rounded][shadow=false]
\setbeamertemplate{title page}[default][colsep=-4bp, rounded=true, shadow=false]
\makeatother

% Title page
\begin{document}

\section{Bevezetés}
\title[]{Üzleti Intelligencia}
\subtitle{Tantárgyi útmutató}
\author[Kuknyó Dániel]{Kuknyó Dániel\\Budapesti Gazdasági Egyetem}
\date{2023/24\\1.félév}
\makebeamertitle

\begin{frame}{Tartalom}
\tableofcontents{}
\end{frame}


\begin{frame}{A félév tematikája}

	\begin{itemize}
		\item[-] Bevezetés
		\begin{itemize}
			\item[\textbf{1.}] \textbf{Verziókezelés}: Git tárhely, alapvető Git parancsok
		\end{itemize}
		\item[-] Megerősítéses tanulás
		\begin{itemize}
			\item[\textbf{2.}] \textbf{Bevezetés a megerősítéses tanulásba}: alapfogalmak, értékfüggvények
			\item[\textbf{3.}] \textbf{Markov döntési folyamatok megoldása}: Mohó ügynök, dinamikus programozás
			\item[\textbf{4.}] \textbf{Sztochasztikus becslés}: Monte Carlo, temporális differenciálás
			\item[\textbf{5.}] \textbf{Q-tanulás}: $Q$-tanulás, Dupla $Q$-tanulás,
			\item[\textbf{6.}] \textbf{Megerősítéses mélytanulás}: $Q$-hálózatok, dupla $Q$-hálózatok, párbajozó $Q$-hálózatok, aktor-kritikus architektúra
		\end{itemize}
		\item[-]{Mélytanulás}
		\begin{itemize}
			\item[\textbf{7.}] \textbf{Transzfertanulás}
			\item[\textbf{8.}] \textbf{Autoencoder architektúrák}
			\item[\textbf{9.}] \textbf{Objektum detekció}
			\item[\textbf{10.}] \textbf{Visszacsatolásos neurális hálózatok}
			\item[\textbf{11.}] \textbf{Transzformáló archtiketkúrák}
		\end{itemize}
	\end{itemize}
	
\end{frame}

\section{Követelmények}

\begin{frame}{Ponthatárok}
	
\end{frame}


\end{document}