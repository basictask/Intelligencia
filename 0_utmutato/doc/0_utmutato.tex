% Settings for the default beamer theme
\documentclass[english, aspectratio=169]{beamer}
\usepackage[T1]{fontenc}
\usepackage[utf8]{inputenc}
\usepackage{tabularx}
\usepackage{babel}
\setcounter{secnumdepth}{3}
\setcounter{tocdepth}{3}

\makeatletter

\newcommand\makebeamertitle{\frame{\maketitle}}

% (ERT) argument for the TOC
\AtBeginDocument{%
  \let\origtableofcontents=\tableofcontents
  \def\tableofcontents{\@ifnextchar[{\origtableofcontents}{\gobbletableofcontents}}
  \def\gobbletableofcontents#1{\origtableofcontents}
}

% Theme settings
\usetheme{Frankfurt}
\usecolortheme{default}
\usefonttheme[onlymath]{serif}

% Template settings
\setbeamertemplate{navigation symbols}{}
\setbeamertemplate{blocks}[rounded][shadow=false]
\setbeamertemplate{title page}[default][colsep=-4bp, rounded=true, shadow=false]
\makeatother

% Title page
\begin{document}

\section{Bevezetés}
\title[]{Üzleti Intelligencia}
\subtitle{Tantárgyi útmutató}
\author[Kuknyó Dániel]{Kuknyó Dániel\\Budapesti Gazdasági Egyetem}
\date{2023/24\\1.félév}
\makebeamertitle

\begin{frame}{Tartalom}
\tableofcontents{}
\end{frame}

\begin{frame}{Elérhetőségek}
\begin{center}
\begin{itemize}
	\item \textbf{E-mail}: \href{mailto:daniel.kuknyo@mailbox.org}{daniel.kuknyo@mailbox.org}
	\item \textbf{Messenger}: \href{https://www.facebook.com/dani.kkny/}{Dani Kuknyo}
	\item \textbf{Coospace üzenet}
	\item Teams-en nem vagyok rendszeresen elérhető. 
\end{itemize}
\end{center}
A fenti címeken lehetősége van minden hallgatónak kérdezésre és konzultációt egyeztetni. A konzultáció platformja Teams, előre megbeszélt időpontban.
\end{frame}

\begin{frame}{A félév tematikája}

	\begin{itemize}
		\item[-] Bevezetés
		\begin{itemize}
			\item[\textbf{1.}] \textbf{Verziókezelés}
		\end{itemize}
		\item[-] Megerősítéses tanulás
		\begin{itemize}
			\item[\textbf{2.}] \textbf{Bevezetés a megerősítéses tanulásba}
			\item[\textbf{3.}] \textbf{Markov döntési folyamatok megoldása}
			\item[\textbf{4.}] \textbf{Sztochasztikus becslés}
			\item[\textbf{5.}] \textbf{Q-tanulás}
			\item[\textbf{6.}] \textbf{Megerősítéses mélytanulás}
		\end{itemize}
		\item[-]{Mélytanulás}
		\begin{itemize}
			\item[\textbf{7.}] \textbf{Bevezetés a mesterséges mélytanulásba}
			\item[\textbf{8.}] \textbf{Objektum detekció}
			\item[\textbf{9.}] \textbf{Egyed szegmentáicó}
			\item[\textbf{10.}] \textbf{Visszacsatolásos neurális hálózatok}
			\item[\textbf{11.}] \textbf{Transzformáló archtiketkúrák}
		\end{itemize}
	\end{itemize}
	
\end{frame}

\section{Követelmények}

\begin{frame}{Követelmények}
\begin{itemize}
	\item A félév során \textbf{2 gyakorlati beadandót} kell teljesíteni, egyet megerősítéses tanulás és egyet mesterséges mélytanulás témaköréből. 
	\item A beadandók \textbf{egyéniek, és hallgatónként más algoritmusokat kell implementálni}.
	\item Egyenként 50-50 pont elérhető. Ez \textbf{összesen 100} gyakorlati pont. 
	\item \textbf{Az egyedi munka elvárt és ellenőrzött}. Plágium esetén a munka érvénytelen lesz. 
\end{itemize}
\par\smallskip
\begin{columns}
\begin{column}{.3\textwidth}
\end{column}
\begin{column}{.3\textwidth}
\begin{block}{\begin{center}Ponthatárok\end{center}}
\begin{center}
$90 \leq p < 100 \Rightarrow 5$\\
$80 \leq p < 90 \Rightarrow 4$\\
$70 \leq p < 80 \Rightarrow 3$\\
$60 \leq p < 70 \Rightarrow 2$\\
$p < 60 \Rightarrow 1$
\end{center}
\end{block}
\end{column}
\begin{column}{.3\textwidth}
\end{column}
\end{columns}
\end{frame}

\begin{frame}{Beadás menete}
\begin{itemize}
	\item A félév során mindenkinek létre kell hoznia egy \textbf{saját Git tárhelyet}, ahol a féléves munkáját fogja rögzíteni:
	\begin{itemize}
		\item A Git felhasználónév teljesen mindegy, de a név mezőbe a teljes nevet írjátok be. 
		\item A tárhely legyen privát. 
		\item Vegyetek fel engem, mint hozzájáruló fejlesztőt a tárhelyre \emph{basictask} felhasználónévvel. 
	\end{itemize}
	\item A munkák beadása Coospace felületen történik. \textbf{Csak egy linket várok, ami a beadandó feladathoz tartozó Git tárhelyre mutat}. Fájlokat és egyéb állományokat nem lehet feltölteni Coospace-re.
	\item \textbf{Késői beadásra nincs lehetőség}. Ha a határidő után történik mentés a tárhelyre, nem lesz figyelembe véve.
\end{itemize}
\end{frame}












\end{document}