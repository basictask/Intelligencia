% Settings for the default beamer theme
\documentclass[english, aspectratio=169]{beamer}
\usepackage[T1]{fontenc}
\usepackage[utf8]{inputenc}
\usepackage{tabularx}
\usepackage{babel}
\setcounter{secnumdepth}{3}
\setcounter{tocdepth}{3}

\makeatletter

\newcommand\makebeamertitle{\frame{\maketitle}}

% (ERT) argument for the TOC
\AtBeginDocument{%
  \let\origtableofcontents=\tableofcontents
  \def\tableofcontents{\@ifnextchar[{\origtableofcontents}{\gobbletableofcontents}}
  \def\gobbletableofcontents#1{\origtableofcontents}
}

% Theme settings
\usetheme{Frankfurt}
\usecolortheme{default}
\usefonttheme[onlymath]{serif}

% Template settings
\setbeamertemplate{navigation symbols}{}
\setbeamertemplate{blocks}[rounded][shadow=false]
\setbeamertemplate{title page}[default][colsep=-4bp, rounded=true, shadow=false]
\makeatother

\begin{document}

% Title page
\section{Bevezetés}
\title[]{Üzleti Intelligencia}
\subtitle{1. Előadás: Verziókezelés és dokumentálás}
\author[Kuknyó Dániel]{Kuknyó Dániel\\Budapesti Gazdasági Egyetem}
\date{2023/24\\1.félév}
\makebeamertitle

% Table of contents slide
\begin{frame}{Tartalom}
\tableofcontents{}
\end{frame}

\section{Verziókezelés alapfogalmai}

\begin{columns}

\begin{column}{0.5\textwidth}
\begin{frame}{Verziókezelés alapjai}
	Miért van szükség verziókezelésre?
	\begin{itemize}
		\item A program változásainak követése
		\item A munka biztonságos elmentése
		\item Kollaboráció több fejlesztő között
		\item Programkód párhuzamos szerkesztése
		\item Feladatok szétosztása és követése
	\end{itemize}
\end{frame}
\end{column}

\begin{column}{0.5\textwidth}
	
\end{column}

\end{columns}

\end{document}



















